\documentclass{article}

\begin{document}
\section{Hardware Properties of the Memory Device}
\begin{tabular}{l c r}
    \hline
    \textbf{Information} & \textbf{Offset (decimal)} & \textbf{Value}\\
    \hline
    Device name & 43 & "NO NAME    "\\
    Serial number & 39 & 464598564\\
    Filesystem type & 54 & "FAT12   "\\
    System identifier & 3 & MSDOS5.0\\
    Media descriptor & 21 & 0xF0 (floppy)\\
    Bytes per sector & 11 & 2\\
    Number of reserved sectors & 14 & 256\\
    Number of sectors per allocation & 13 & 1\\
    Number of sectors per FAT & 22 &2304\\
    Size of the device (bytes) & ? & 179184\\
    Number of sectors per track & 24 & 4608\\
    Number of heads or sides on the diskette & 26 & 512\\
    Number of hidden sectors & 28 & 0\\
    Start of Bootstrap routine & 1 &60\\
    Number of FAT & 16 & 2\\
    Offset to start of FAT(s) & - & 512\\
    BIOS boot Signature & 510 & 0x55AA\\
    Root Directory Offset & - & 9728\\
    Offset to data area & - & 16896\\
    Additional Information & &\\
    of interest you find & &\\
    \hline
\end{tabular}\\
\\
\\
Note: Seems like the bytes 11 and 12 got mixed up. If they were in reverse order the sector size would correctly be 512 bytes, instead it is 2 bytes.

\section{FAT Investigation}

Find out where the virus has corrupted the FAT tables and suggest a way to correct it.

\section{Investigation of Directories}

For each directory fill one of these:\\
\\
\begin{tabular}{l c r}
    \hline
    \textbf{Information} & \textbf{Offset (Size)} & \textbf{Value}\\
    Directory/File Name & ? & ? \\
    Attributes & ? & ? \\
    Creation Time and Date & ? & ? \\
    Last Access Date & ? & ? \\
    Time and Date Stamp & ? & ? \\
    Cluster’s Chain in FAT & ? & ? \\
    Absolute Offset & ? & ? \\
    Size of the file & ? & ? \\
    \hline
\end{tabular}
\\
\\
\\
Also find and analyze the anomalies.
\section{Attack on Zip-archive}

\end{document}
