\documentclass{article}

\begin{document}

\begin{tabular}{|l|r|r|}
    \hline
     & Offset (decimal) & value\\
    \hline
    Device name & 43 & "NO NAME    "\\
    \hline
    Serial number & 39 & 464598564\\
    \hline
    Filesystem type & 54 & "FAT12   "\\
    \hline
    System identifier & 3 & MSDOS5.0\\
    \hline
    Media descriptor & 21 & 0xF0 (floppy)\\
    \hline
    Bytes per sector & 11 & 2\\
    \hline
    Number of reserved sectors & 14 & 256\\
    \hline
    Number of sectors per allocation & 13 & 1\\
    \hline
    Number of sectors per FAT & 22 &2304\\
    \hline
    Size of the device (bytes) & ? & 179184\\
    \hline
    Number of sectors per track & 24 & 4608\\
    \hline
    Number of heads or sides on the diskette & 26 & 512\\
    \hline
    Number of hidden sectors & 28 & 0\\
    \hline
    Start of Bootstrap routine & 1 &60\\
    \hline
    Number of FAT & 16 & 2\\
    \hline
    Offset to start of FAT(s) & - & 512\\
    \hline
    BIOS boot Signature & 510 & 0x55AA\\
    \hline
    Root Directory Offset & - & 9728\\
    \hline
    Offset to data area & - & 16896\\
    \hline
    Additional Information & &\\
    of interest you find & &\\
    \hline
\end{tabular}

Seems like the bytes 11 and 12 got mixed up. If they were in reverse order the sector size would correctly be 512 bytes, instead it is 2 bytes.

\end{document}
